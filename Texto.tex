% \documentclass[a4paper, 12pt]{article}

% UTF 8 encoding
\usepackage[utf8]{inputenc}

% Brazilian Portuguese
\usepackage[brazil]{babel}

% Margins of the page
\usepackage{geometry}[left=3cm, right=2cm, top=3cm, bottom=2cm]

% Metadados
\title{Teste Miktex no VS Code}
\author{Luís Henrique Araújo Couto}
% data atual
\date{\today}

% Configuração do título
\makeatletter
\renewcommand{\maketitle}{
    \begin{center}
        {\huge\textbf{\@title}}
        \par\smallskip\noindent
        \begin{tabular}[t]{c}
            \@author
        \end{tabular}
        \par\smallskip\noindent
        \begin{tabular}[t]{c}
            \@date
        \end{tabular}
    \end{center}
}

% import de multilinhas tabelas
\usepackage{multirow}


% Inicio do documento
\begin{document}

\documentclass[a4paper,12pt]{article}

% utf 8 encoding
\usepackage[utf8]{inputenc}

% portuguese brazil
\usepackage[brazil]{babel}

% use geometry
\usepackage{geometry}
\geometry{a4paper, left=3cm, right=2cm, top=3cm, bottom=2cm}

% use graphicx
\usepackage{graphicx}

% use setspace
\usepackage{setspace}


\begin{document}
    
%================================================================
% dados instituição, disciplina e professor
%================================================================

\begin{titlepage}
    \begin{center}

        \includegraphics[width=0.25\textwidth]{imagens/logoifba.png}\\
        %coloque o logo dentro da pasta logo, e coloque o nome o diretorio e nome do arquivo com extensão dentro dos {}, ajuste o tamanho do logo no "width"
        
        INSTITUTO FEDERAL DE EDUCAÇÃO, CIÊNCIA E TECNOLOGIA DA BAHIA\\
      
        ANÁLISE E DESENVOLVIMENTO DE SISTEMAS\\
        
        DISCIPLINA: LABORATÓRIO DE METODOLOGIA CIENTÍFICA\\
        
        DOCENTE: JULIO CESAR\\
        
%================================================================
% dados matéria, autor e data
%================================================================

        \vfill %serve para alinhar essa parte no centro vertical da folha
        
        \textbf {APRENDENDO LaTeX NO MODO JUSCELINO KUBITSCHEK}\\
        
        LUÍS HENRIQUE ARAÚJO COUTO % NOME
        %caso haja mais de um autor, colocar o nome completo com \\ no final
        
        \vfill
        
        EUNÁPOLIS - BA, 01 DE ABRIL DE 2024 % DATA
        
    \end{center}
\end{titlepage}

\end{document}


\maketitle

% texto

\section{Introdução}
Este é apenas um texto de teste do VS Code como interpretador do Miktex/Latex, para a disciplina de Laboratório de Metodologia Científica. Além disso, muito provavelmente este mesmo interpretador personalizado vai ser usado para o desenvolvimento de texto acadêmicos mais abrangentes usados e desenvidos pelo autor.

\section{Objetivos}
Aprender essa matéria o mais rápido possível dentro de uma semana para que na prova, eu não só consiga fazer bons arquivos, mas também de maneira rápida e eficiente para isso posssa resultar em uma boa (7) ou ótima (10) nota.

\section{Metodologia}
A Metodologia de estudo vai no estilo Jucelino Kubitschek, 40 assuntos em apenas 4 dias. Como dizia os Racionais, "o bagulho é louco e o processo é lento, mas o resultado é foda e o final é violento". Portanto, vamos não só conseguir alcançar o objetivo, mas também evitar o desastre de ir para prova final sem saber absolutamente nada.

% tabela de objetivos
\begin{table}[h]
    \centering
    \begin{tabular}{|c|c|}
        \hline
        \textbf{Objetivo} & \textbf{Descrição} \\
        \hline
        Objetivo 1 & Aprender a usar o Miktex no VS Code \\
        \hline
        Objetivo 2 & Fazer um texto acadêmico \\
        \hline
        Objetivo 3 & Aprender a usar o Miktex no VS Code \\
        \hline
        Objetivo 4 & Fazer um texto acadêmico \\
        \hline
    \end{tabular}
    \caption{Objetivos do estudo}
    \label{tab:objetivos}
\end{table}

% texto grande e centralizado escrito "FIM", em negrito e com margem afastada do texto acima
\begin{center}
    \textbf{\large{FIM}}
\end{center}


\end{document}